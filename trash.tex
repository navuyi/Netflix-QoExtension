% Architecture - start
\newpage
\section{Architecture}
\label{sec:architecture}
    In order to fulfill our goal, running more ecologically experiments, we had to use everyday service. We started from YouTube, but our preliminary test showed that users are not using YouTube for pure watching \cite{}. The next reasonable service would be any streaming service providing content which is watched as whole, examples are NETFLIX, Disnay+, Amazon Prime, HBO Go, to name a few. We selected Netflix for couple of reasons. It is a popular service in Poland, it provides certain statistics about the streamed content, last but not least, we are able to select specific bitrate provided by the server. In order to fully utilize information and control provided by Netflix we created a Chrome extension/tool? \verb|Netflix QoExtension|. 
    
    The tool consists of three connected components. REST API application, database and Chrome web browser extension being the core module of the entire tool. 
    Separation of concerns guarantees that future developers will have greater flexibility in utilizing various technologies to make the tool suitable for their particular use case.
    
    

    \subsection{Chrome extension}
        Chrome extension is the core module of the entire tool. It allows for measuring objective video playback metrics, asking subjects to score video quality, manipulating video quality according to predefined scenario and more. Most important features are described in detail in \ref{sec:Features}.
        Furthermore, extension includes dedicated user interface (UI) in form of custom web pages allowing experimenters prepare and conduct experiments seamlessly.
        The technology stack includes basic web technologies like Javascript, HTML, CSS and additional frameworks and tools like React, Webpack and Typescript.

        Interaction with Netflix web application is possible due to the extension listening for changes in browser's URL address bar. 
        When loaded URL matches certain pattern and Netflix video player page is detected a \textit{content script} is injected into the page. 
        It has the ability to access DOM tree of the original Netflix page, modify them, delete and even create and add custom elements. 

        Content script is executed in isolated environment meaning that it cannot access variables and functions defined in original Netflix script including the \textit{netflix} object being the entry point to Netflix API.
        Without that access many features are not available including video seeking.
        However a workaround was implemented resulting in full access to the \textit{netflix} API.
        
    \subsection{REST API}
        Back-end side of the tool consist of REST API (from this point forward, referred as server) connected to a database. Server receives requests with data captured by the extension, processes it and stores in the database. Other than that, server also sends information such as database ID of currently played video to the extension in order to assign captured data accurately.
        Server operates locally at the same machine the extension is running on, however it was designed in a way that migrating to a remote server is simply a matter of networking and minor tweaks in the code.
        
        Server was implemented using Python and Flask micro-framework.
    
    \subsection{Database}
        Relational database was selected due to its flexibility and ease of retrieving desired information. 
        The tool utilizes SQLite3 database because it's file based which allows for quick and simple configuration.
% Architecture - end



% Features - start
\section{Features}
\label{sec:Features}
    This part focuses mainly on the Chrome extension part of the tool as it contains most of important features.
    
    \subsection{Video quality manipulation}
        Extension exploits a built-in feature - \textit{bitrate menu}, accessible in Netflix thanks to work of \cite{netflix-1080p} and their modifications of \textit{Cadmium Playercore} being the Netflix's video player implementation in Javascript, which allows for manipulation of audio and video bitrate and selection of Content Delivery Network (CDN) - server the content will be served from. 
        
        The possibility of changing video bitrate is used numerous times throughout experiment. 
        Extension programatically simulates special keyboard key event (CTRL+ALT+SHIFT+S) to invoke bitrate menu, gains access to required HTML elements using DOM tree searching and selects video bitrate according to a value being a compromise between scenario provided through configuration file and bitrate values available for particular video. 

    
    \subsection{Debug Menu data}
        Debug Menu is also a built-in feature but in opposite to bitrate menu it does not require any modifications of CadmiumPlayercore and is accessible using keyboard CTRL+SHIFT+ALT+D key event.
        This feature is accessible by default and is used to gather information about state of video playback such as current audio/video bitrate, vmaf, volume or buffering audio/video bitrate and vmaf and more. Content script injected into the page screens HTML DOM tree of the page in order to find HTML element displaying desired data. Value of the element is analyzed using regular expressions in order to filter out required information.

        
    \subsection{Video quality scores}
        Quality scores are obtained by displaying an assessment panel every defined time interval. Assessment panel covers entire screen completely blocking out video which for the time of scoring is paused. Panel consists of a question to score video quality and five buttons representing 1-5 MOS scale \cite{P.800}. After scoring quality video playback automatically resumes. Time of scoring is tracked and captured by the software.
    
    \subsection{Bitrate-VMAF mapping mode}
        The tool cannot explicitly tweak Netflix player to display video with defined VMAF but it does so by manipulating video bitrate which is directly related to VMAF values. Each available bitrate corresponds to one VMAF value. VMAF values displayed in Debug Menu are average values for given video chunk encoded with particular bitrate.
    
        Bitrate to VMAF mapping is a process of creating relation between bitrates and VMAF values available for particular video in Netflix streaming platform. It requires \textit{empty} configuration file containing \textit{template VMAFs} being a list of VMAF values intended to be displayed during video playback. In the process bitrate to VMAF mappings are discovered and \textit{complete} configuration file is created. Complete configuration consist of a list of VMAF values being closest to values from template and their bitrate substitutes.

            
    \subsection{Main mode}
        Mapping mode can be used to generate \textit{complete} configuration files that will be utilized in main mode. Graphical user interface makes it seamless to prepare and start experiment. After providing configuration file and clicking start button software immediately navigates to the first video specified in scenario. After loading video player all bitrate changes and quality assessment popups are scheduled to appear in defined time intervals. Video always starts from beginning. 
    
        During video playback subjects are restricted from certain actions within player such as: video seeking and pausing. Some of the original Netflix user interface components are removed or modified by the extension in order to enforce restrictions.
        Bitrate changes are executed according to the generated scenario and configuration file. End of video is detected by checking for characteristics HTML element such as "See credits" button which appears near the end of every video in a TV series. After video end is detected tool redirects page to custom extension page that comes in two cases depending if there are more videos to be displayed. If there are videos left a short break can be made where subject decides when to continue by clicking \textit{Continue} button. If experiment has finished proper information is presented to the subject.

% Features - end